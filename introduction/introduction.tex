\section{Introduction}

\emph{Ubiquitous Computing} (synonymous with \emph{Pervasive Computing}) is the area of research concerning
ever-accessible and ever-present computers. The term was coined in 1991 by Mark Weiser, and this original
definition contained several more details than those immediately understandable from the name.

Ubiquitous Computing regards motion-, temperature-, and other sensors; timers and clocks; digital travel
cards;\footnote{Such as the Danish \emph{Rejsekort}, the \emph{Oyster cards} found in London, and many more
all around the globe.} and other technology that does not dominate, but rather facilitates normal human
behaviour. As Weiser puts it, ``ubiquitous computing forces the computer to live out here in the world with
people.''\cite{weiseronline} He goes on to provide further detail about the number of computers required for
this state of ubiquity to be reached:

\begin{quote}
     Activate the world. Provide hundreds of wireless computing devices per person per
     office, of all scales (from 1'' displays to wall sized). [...] It is invisible, everywhere computing that does
     not live on a personal device of any sort, but is in the woodwork everywhere.\cite{weiseronline}
\end{quote}

Weiser provides five qualities that are required for Ubiquitous Computing. The computers must be
\textbf{many}, \textbf{small}, \textbf{everywhere}, \textbf{invisible}, and \textbf{single-purpose}. That is to say,
not only must the computers be ever-accessible, they must also be split into small components, each geared towards
a specific task.

The most easily overlooked quality of Weiser's vision is the single-purpose focus of all computers. He discusses
the possibility of computers replacing traditional, physical entities, going from ``wall notes, titles on book spines,
labels on controls, thermostats and clocks, as well as small pieces of paper'' to ``more than 100 tabs, 10 or 20 pads
and one or two boards.''\footnote{Tabs are the smallest computers which are used through an interactive surface, pads
are slightly larger, and boards the largest. All of these were developed at Xerox PARC at the time of Weiser's original
writing.}\cite{weiser91} His vision for the future is one where even personal computing is overtaken by the ubiquitous
kind.

\subsection*{``Dirty'' Ubiquitous Computing}

The original paper described a vision of a near-future in which computing would reach the \emph{ubiquitous} state.
This rhetoric of \emph{near-future} was carried on in other papers regarding this subject, eventually leading Bell
and Dourish to write a response.

Bell and Dourish contest the original definition, urging researchers of Ubiquitous Computing to accept reality, not
prophesize about some potential future. They argue that some state of ubiquitous computing has already been
reached, albeit a messier and less obvious kind than the one originally anticipated.

\textbf{Note: Details of Bell \& Dourish definition as well as where it differs from original. Mention why it is "dirty"
(they say this).\cite{bell07}}

It is a noble cause to want qualified researchers to spend their time doing actual research, as opposed to
considering entirely hypothetical situations.\footnote{Determining whether or not researchers in the field of
Ubiquitous Computing were actually, until 2007, wasting everyone's time is outside of the scope of this paper,
and as such, Bell and Dourish shall be given the benefit of the doubt.} Bell and Dourish's motivation seems to
be to straighten out a group a researchers, making sure they do good work. The change of the very definition of
Ubiquitous Computing is, to them, nothing more than (necessary) collateral damage.

However noble the motivations of the paper, it ended up causing more harm than good. The Ubiquitous Computing described
by Bell and Dourish differs so much from previous definitions that it is entirely unrecognizable and should be considered
an entirely different thing, and should not have been introduced as Ubiquitous Computing.

Changing the definition of an existing term is inherently problematic. Ubiquitous Computing was previously defined
as a certain state of computing, towards which we were moving. It was very clear whether or not this state had been
reached. (In fact, I provided a checklist above, based on Weiser's definition, that should be handy when trying to
either refute or confirm that the state has been reached.)

The new definition is a watering down of the original, allowing for the present day's computing to be called ubiquitous.
This kind of change looks like magic to the uninitiated outsider: we are looking towards a potential future,
thinking that one day we might get truly Ubiquitous Computing. Suddenly, out of nowhere, we are already there.
Granted, it's not quite what we were looking forward to, but \emph{it's something}.

The absurdity might be best explained by metaphor. I like pancakes. My girlfriend has promised me pancakes for dinner,
but when she arrives home she has not brought anything. I am slightly disappointed, but she hurries to explain
herself: she changed the definition of pancakes to be any kind of glucose-heavy food that would possibly fit on a pan.
She then proceeds to serve the stale bread from the bread basket, calling it pancakes.

These were obviously not the pancakes we had agreed on, and in the same way that I feel cheated, the entire group of researchers concerning
themselves with Ubiquitous Computing should feel cheated. They have all been talking about Ubiquitous Computing, knowing
what it was, why present day computing wasn't ubiquitous, how far along the process of getting there they were, when suddenly
Ubiquitous Computing is already here. It just so happens that it's different from what they're talking about: It is
not really everywhere; it is not really small; it is not really invisible; and it is not single-purpose, either. There are
a lot of computers, but not nearly as many as they had expected.

Having several definitions of the same term causes problems when trying to discuss the merits of either of them: when I say
that Ubiquitous Computing is undesirable (as I will in Section \ref{sec:devices}), how will
an audience know which of the two definitions I am addressing?

As should be evident, I do not agree with the new definition of Ubiquitous Computing. As such, from this line onward,
\emph{Ubiquitous Computing} shall refer to Ubiquitous Computing as defined by Weiser in 1991, and Bell and Dourish's
definition shall be known as \emph{Fairly-Present Computing}, or \emph{What We Happen To Have Today}.