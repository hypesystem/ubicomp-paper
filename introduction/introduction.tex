\section{Introduction}

\emph{Ubiquitous Computing} (synonymous with \emph{Pervasive Computing}) is the area of research concerning
ever-accessible and ever-present computers. The term was coined in 1991 by Mark Weiser, and this original
definition contained several more details than those immediately derivable from the name.

Note: Detail these\cite{weiser91}

\begin{quote}
     Our preliminary approach: Activate the world. Provide hundreds of wireless computing devices per person per
     office, of all scales (from 1" displays to wall sized). [...] It is invisible, everywhere computing that does
     not live on a personal device of any sort, but is in the woodwork everywhere. 
     
    (\verb+http://www.ubiq.com/hypertext/weiser/UbiHome.html+)
\end{quote}

The original paper described a vision of a near-future in which computing would reach the "ubiquitous" state.
This rhetoric of "near-future" was carried on in other papers\footnote{source(s)?} regarding this subject,
eventually leading Bell and Dourish to write a response.

Bell and Dourish contest the original definition, urging the researchers of the field to accept reality, not
prophesize about some potential future. They argue that some state of ubiquitous computing has already been
reached, albeit a messier and less obvious kind than the one originally anticipated.\cite{bell07}

Note: Details of Bell \& Dourish definition as well as where it differs from original.

Note: Several good reasons for them urging the change: researchers were being unscientific, and has nothing to
lose. It was bad science

Note: Ultimately unscientific to simply "change the meaning of a term" to claim that we have it. A better approach
would have been to introduce a new term to mean this new, messy ubicomp.

Note: The new definition for the same term makes discussion of the merits and desirability of the original definition
very challenging. If one is to say "ubicomp" today, nobody would know which of the two we are talking about. This
is damaging for all intents and purposes. The new version is something we have, and therefore something that can be
assumed to be desirable (something we are actually moving towards). The old version, however, is not necessarily so.
Mention why the old one might not be so good (in brief) with references to discussion of these details in the paper.

Note: Introduce terminology used in this paper. Ubiquitous Computing means the original definition; find a term for the
new kind.