\section{Introduction}

\emph{Ubiquitous Computing} (synonymous with \emph{Pervasive Computing}) is the area of research concerning
ever-accessible and ever-present computers. The term was coined in 1991 by Mark Weiser, and this original
definition contained several more details than those immediately derivable from the name.

Ubiquitous Computing regards motion-, temperature-, and other sensors; timers and clocks; digital travel
cards;\footnote{Such as the Danish \emph{Rejsekort}, the \emph{Oyster cards} found in London, and many more
all around the globe.} and other technology that does not dominate, but rather facilitates normal human
behaviour. As Weiser puts it, "ubiquitous computing forces the computer to live out here in the world with
people."\cite{weiseronline}

Note: Briefly the things weiser says: many, small, everywhere, single-purpose\cite{weiser91}

\begin{quote}
     Our preliminary approach: Activate the world. Provide hundreds of wireless computing devices per person per
     office, of all scales (from 1" displays to wall sized). [...] It is invisible, everywhere computing that does
     not live on a personal device of any sort, but is in the woodwork everywhere.\cite{weiseronline}
\end{quote}

The original paper described a vision of a near-future in which computing would reach the "ubiquitous" state.
This rhetoric of "near-future" was carried on in other papers regarding this subject, eventually leading Bell
and Dourish to write a response.

Bell and Dourish contest the original definition, urging the researchers of the field to accept reality, not
prophesize about some potential future. They argue that some state of ubiquitous computing has already been
reached, albeit a messier and less obvious kind than the one originally anticipated.

Note: Details of Bell \& Dourish definition as well as where it differs from original.\cite{bell07}

Note:
Other examples of watering down the definition? (Layers)
\verb+http://en.wikipedia.org/wiki/Ubiquitous_computing#Core_concepts+,
\verb+http://en.wikipedia.org/wiki/The_Information_Age:_Economy,_Society_and_Culture#The_Rise_of_the_Network_Society+,
\verb+http://intel-research.net/Publications/Seattle/090520031604_152.pdf+
Other critisisms/problems that need to be overcome for ubicomp to become a reality
\verb+http://link.springer.com/chapter/10.1007/3-540-45427-6_22#page-1+
\verb+http://wiki.daimi.au.dk/OT-intern/_files/cacm-ubicomp.pdf+

It seems a noble cause to want qualified researchers to spend their time doing actual research, as opposed to
considering entirely hypothetical situations.\footnote{Determining whether or not researchers in the field of
Ubiquitous Computing were actually, until 2007, wasting everyone's time is outside of the scope of this paper,
and as such, Bell and Dourish shall be given the benefit of the doubt.} Bell and Dourish's motivation seems to
be to straighten out a group a researchers, making sure they do good work. The change of the very definition of
Ubiquitous Computing is nothing more than (necessary) collateral damage.

However noble the motivations of the paper, the collateral damage it does is more costly than anything it may have
achieved. It impacts discussions in the field of Ubiquitous Computing as a whole. The Ubiquitous Computing described
by Bell and Dourish differs so much from previous definitions that, I would argue, it is an entirely different
thing. The problem arises when trying to discuss the merits of the previous definition: when I say that Ubiquitous
Computing is not only infeasible, but also undesirable (as I will in Section \ref{trollolo}), how will an audience
know which of the two definitions I am addressing?

The change of definition, in itself, is problematic, too. Ubiquitous Computing was \emph{defined} as a certain state
of computing, towards which we were moving. It was very clear whether or not this state had been reached. (In fact,
I provided a checklist above, based on Weiser's definition, that should be handy when trying to refute or confirm that
the state has been reached.)

Changing the definition is changing everything. All of a sudden we are no longer looking
for something, we have it. This could be likened to discussing pancakes. I like pancakes. Now, my girlfriend has
promised me pancakes for dinner, but when she arrives home she has brought muffins. I am slightly disappointed (although
it \emph{is} still cake), but she hurries to explain herself: she changed the definition of pancakes to be any kind
of cake that would possibly fit on a pan. This is obviously not what we had agreed on, and in the same way, the entire
group of researchers concerning themselves with Ubiquitous Computing should feel cheated. They have all been talking
about Ubiquitous Computing, knowing what it was, why it wasn't here yet, how far along the process of getting there
they were, when suddenly they are told that Ubiquitous Computing is already here. It just so happens that it's different
from what they're talking about. It's not really everywhere, it's not really small, but it's kind of everywhere and kind
of small.

As should be evident, I do not agree with the new definition of Ubiquitous Computing. As such, from this line onward,
\emph{Ubiquitous Computing} shall mean \emph{Ubiquitous Computing as defined by Weiser in 1991}, and Bell and Dourish's
definition shall be known as \emph{Fairly-Present Computing}.

Note: Mention why the old one might not be so good (in brief) with references to discussion of these details in the paper.