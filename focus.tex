\section{Conclusion}

I would disagree with Bell and Dourish's claim that we have Ubiquitous Computing all around us. That said, I would not disagree
with their point that we, as researchers, and when reflecting on IT in general, should work with what exists in the world, the tendencies we actually see, rather than some vision of
a could-be future.

We are not seeing a tendency towards more ubiquity in computers, except in a few areas, but rather towards an aggregation of
functionality in personal computers, both portable and stationary. Weiser predicted that Personal Computing would peak around
2004, but the oncoming of portable personal computers made this prediction false.

In this paper, I suggested to use the term \emph{devices} to best describe the reality of computing today. Devices can offer a
great degree of functionality, both in traditional un-ubiquitous kind, such as office tools, multimedia suites, etc;
and of a kind that is almost ubiquitous, such as GPS tracking, body signal measurement, etc.

Devices offer a great level of inter-connectivity, many synchronizing data across devices. As such, customers have a copy of their
most used data on each device, whether at home, at the office, or on the move.

The popularity of devices also has a lot to do with the use of them being entirely voluntary. Whereas in a world with truly ubiquitous
computing it would be hard to escape the computers, devices are easy to put down. The ability to choose whether or not to use the
technology makes many more people choose it.

While I have chosen to neither introduce yet another definition of Ubiquitous Computing, nor a new all-encompassing term,
I do believe that a term adequately describing the state of computing today can and should be found.