\section{Conclusion}

Researchers of ubiquitous computing have been chasing fairies\footnote{Or have they? Sources.} by trying to reach
a point where small single-purpose computers exist. They need to embrace what is happening: devices.

(Maybe they already have? Check this...)

Maybe they should use a different term than ubiquitous computing? Pros and cons of this?

Note: Maybe it would be a viable way to start a new ubicomp idea directly as integrated with a smartphone? Some, like FitBit,
do this to some extent, letting the smartphone gather data. It is hard for some things, like FitBit, because existing devices
don't have the necessary sensors. It is very possible that it requires a well-established brand, like Apple's, to successfully
launch an entirely new type of device.