\section{Devices}

When describing his vision for Ubiquitous Computing, Weiser draws a parallel to the evolution of the electric engine:

\begin{quote}
    How do technologies disappear into the background? The vanishing of electric motors may serve as an instructive example.
    At the turn of the century, a typical workshop or factory contained a single engine that drove dozens or hundreds of
    different machines through a system of shafts and pulleys. Cheap, small, efficient electric motors made it possible first
    to give each tool its own source of motive force, then to put many motors into a single machine.\cite{weiser91}
\end{quote}

The electric engine 

Weiser's electrical engine is very much what he describes: many, small, everywhere, single purpose. And to some extent
this is what we see. We have seen some things becoming smaller, more everpresent, and with a single purpose. Sensors, timers,
clocks on everything (microwave, oven, fridge, TV). But what we haven't seen is single-purpose: smartphones, TVs, fridges,
microwaves, etc. all surve multiple purposes. Granted, sensors might be the only exception to this. But that might just 
be because of time constraint. We used to have kitchen timers, nowadays they are built into most appliances (oven).

Note: Introduce the derm \emph{Device}, as a single item holding several processors/computers and other things (sensors, chips,
etc) that cooperate. This is \emph{like} the electrical engine, but \emph{unlike} small computers everywhere!

Devices also result in the computers being less "everywhere" as in "inescapable" and more "everywhere" as in "always accessible".
Choice is king.

While we are seeing "smaller computers" (several processors in one device) we are not seeing them split up over several places,
but rather put together in one handy device. I would conclude that Weiser made an erroneous conclusion.

More and more things become smartphones. Chips become embedded, there are projects to remove credit cards and instead use
smartphones, etc.

We want to be able to take things with us. We want to be able to put things down. This is what makes it undesirable.

Weiser's description of the computers of the day in 1991 could fit those of today, too. We are not moving away, in fact we are
seeing more of this:

\begin{quote}
    Today's multimedia machine makes the computer screen into a demanding focus of attention rather than allowing it to fade
    into the background.\cite{weiser91}
\end{quote}

\begin{quote}
    Doors open only to the right badge wearer, rooms greet people by name, telephone calls can be automatically forwarded to wherever the recipient may be, receptionists actually know where people are, computer terminals retrieve the preferences of whoever is sitting at them and appointment diaries write themselves. The automatic diary shows how such a simple task as knowing where people are can yield complex dividends\cite{weiser91}
\end{quote}

We might get several devices (one person has laptop, tablet, smartphone) but each device does everything.

We lose the single-purpose. We lose the invisible (they draw attention). We lose the everywhere (they are in specific places
so we can put them away). To some extent we lose the many, as we seem to stagnate around 3 devices per person.\footnote{Cisco
predicts that there will, in 2017 be "nearly three networked devices per capita". This includes business devices, so 3 is a
high number for a long time.\cite{cisco} Of course, there are also other devices, such as ovens and microwaves. This ends up
with a count of around 30 in a normal household. Well below 100. --- A different survey counts a maxiumum (Switzerland) of
0,86 devices per person in 2005. This is still well within my prediction.\cite{nationmaster}} The only thing we keep is the small.

\paragraph{stuff}

\begin{quote}
    A good tool is an invisible tool. [...] you focus on the task, not on the tool. (\verb+http://www.ubiq.com/hypertext/weiser/ACMInteractions2.html+)
\end{quote}

With this in mind, maybe my use of invisible is wrong? No. Focus is very much on the tool. It is a different experience to write notes on
a smartphone than a pc, and neither disappear. Both have to be sought out.

"multimedia tries to grab attention, the opposite of the ubiquitous computing ideal of invisibility" (\verb+http://www.ubiq.com/hypertext/weiser/UbiCACM.html+)

But all modern devices are multimedia! Woop!