\section{Multi-purpose devices}

\begin{quote}
    -- quote about engines! --
\end{quote}

Weiser's electrical engine is very much what he describes: many, small, everywhere, single purpose. And to some extent
this is what we see. We have seen some things becoming smaller, more everpresent, and with a single purpose. Sensors, timers,
clocks on everything (microwave, oven, fridge, TV). But what we haven't seen is single-purpose: smartphones, TVs, fridges,
microwaves, etc. all surve multiple purposes. Granted, sensors might be the only exception to this. But that might just 
be because of time constraint. We used to have kitchen timers, nowadays they are built into most appliances (oven).

More and more things become smartphones. Chips become embedded, there are projects to remove credit cards and instead use
smartphones, etc.

We want to be able to take things with us. We want to be able to put things down. This is what makes it undesirable.

Note: Both pervasive and ubiquitous computing have the synonym "inescapable", which is not what we are seeing.
\verb+http://www.oxforddictionaries.com/definition/english/pervasive?q=pervasive+
\verb+http://www.oxforddictionaries.com/definition/english/ubiquitous?q=ubiquitous+
Probably move this to the introduction... (as anti-Bell argument).