\section{Devices}

Note: Introduce the derm \emph{Device}, as a single item holding several processors/computers and other things (sensors, chips,
etc) that cooperate. This is \emph{like} the electrical engine, but \emph{unlike} small computers everywhere!

Devices also result in the computers being less "everywhere" as in "inescapable" and more "everywhere" as in "always accessible".
Choice is king.

While we are seeing "smaller computers" (several processors in one device) we are not seeing them split up over several places,
but rather put together in one handy device. I would conclude that Weiser made an erroneous conclusion.

\begin{quote}
    How do technologies disappear into the background? The vanishing of electric motors may serve as an instructive example.
    At the turn of the century, a typical workshop or factory contained a single engine that drove dozens or hundreds of
    different machines through a system of shafts and pulleys. Cheap, small, efficient electric motors made it possible first
    to give each tool its own source of motive force, then to put many motors into a single machine.\cite{weiser91}
\end{quote}

Weiser's electrical engine is very much what he describes: many, small, everywhere, single purpose. And to some extent
this is what we see. We have seen some things becoming smaller, more everpresent, and with a single purpose. Sensors, timers,
clocks on everything (microwave, oven, fridge, TV). But what we haven't seen is single-purpose: smartphones, TVs, fridges,
microwaves, etc. all surve multiple purposes. Granted, sensors might be the only exception to this. But that might just 
be because of time constraint. We used to have kitchen timers, nowadays they are built into most appliances (oven).

More and more things become smartphones. Chips become embedded, there are projects to remove credit cards and instead use
smartphones, etc.

We want to be able to take things with us. We want to be able to put things down. This is what makes it undesirable.

Weiser's description of the computers of the day in 1991 could fit those of today, too. We are not moving away, in fact we are
seeing more of this:

\begin{quote}
    Today's multimedia machine makes the computer screen into a demanding focus of attention rather than allowing it to fade
    into the background.\cite{weiser91}
\end{quote}