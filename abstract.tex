\begin{abstract}

In 2007, Bell and Dourish offered an alternative to the original definition of Ubiquitous
Computing, due to Weiser's 1991 paper. The override of definition
discouraged discussion of the original idea. This paper
separates the new and old definitions of Ubiquitous Computing, and discusses the merits and
desirability of the computers described in the original definition.

The idea of devices is introduced as a concept that gives depth to the discussion of Ubiquitous
Computing, while simultaneously being at odds with the idea of single-purpose computers, an idea
central to the area. Devices are used to explain why truly ubiquitous computing does not exist.

The possible reasons for the popularity of devices are considered, focusing on the notion of users' choice.

Finally, the future of research in Ubiquitous Computing is discussed, and it is proposed
that a more fitting name for the research area might exist and could be found.

\end{abstract}