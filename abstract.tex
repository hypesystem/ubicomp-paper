\begin{abstract}

In 2007, Bell and Dourish offered an alternative to the original definition of Ubiquitous
Computing, due to Weiser's 1991 paper.\cite{bell07}\cite{weiser91} The override of definition
discouraged discussion of the original idea and its desirability. The purpose of this paper is
to separate the new and old definitions of Ubiquitous Computing, and discuss the merits and
desirability of the original.

The idea of Devices is introduced as a concept that gives depth to the discussion of Ubiquitous
Computing, while simultaneously being at odds with the idea of single-purpose computers, an idea
central to the area.

Finally, the future of research in Ubiquitous Computing is discussed, and it is proposed
that a more fitting name for the area of research might exist, and could be found.

\end{abstract}